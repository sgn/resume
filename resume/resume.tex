% (c) 2018 Danh DOAN <congdanhqx@gmail.com>
% vim: et ts=2 sw=2
%
% The template is licensed under the Creative Commons Attribution-ShareAlike 3.0
% Unported License. To view a copy of this license, visit
% http://creativecommons.org/licenses/by-sa/3.0/ or send a letter to
% Creative Commons
% 171 Second Street, Suite 300
% San Francisco, California, 94105, USA.

\documentclass[letterpaper,11pt]{article}
\usepackage[pagesetup]{tucv}
\usepackage[pdfborder={0 0 0}]{hyperref}
\fancyfoot[C]{Danh DOAN - Résumé - Page \thepage}

\NewDocumentCommand\email{m}{\href{mailto:#1}{#1}}
\RenewDocumentCommand\resentry{O{0pt}mm}{
  \begin{tabular*}{0.9\textwidth}[t]{@{\hspace{#1}}
    p{5in-#1}@{\extracolsep{\fill}}p{1.0in}}
    \raggedright #2 & #3
    \tabularnewline
  \end{tabular*}
}
\def\Plus{\texttt{+}}
\def\Minus{\texttt{-}}
\def\cplusplus{C\Plus\Plus}

\begin{document}

% Page heading and name/contact info table
\begin{tabular*}{7in}{l@{\extracolsep{\fill}}r}
  \multicolumn{2}{c}{\textbf{\Large Danh DOAN}}\\
  %Ho Chi Minh city \\ %& \Plus66\Minus66\Minus666\Minus6666\\
  Ho Chi Minh city & \email{congdanhqx@gmail.com}\\
  Vietnam  & \url{https://danh.dev}
\end{tabular*}

\begin{center}
  \Large Software Engineer
\end{center}

% Resume section heading
\resheading{Highlights of Qualifications}
\begin{itemize}
  \item Have critical thinking mindset, solved wide range problems
    with different programming languages included but not limited to
    C, C++, sh, etc...
    %    \resdesc{C}, \resdesc{\cplusplus{}}, \resdesc{sh}, etc...
  \item Worked and gained good understanding with different version control
    systems like git(1), svn(1).
\end{itemize}

\resheading{Software Development Experience}
% Top-level entry for an employer, providing employer, location, and description
% fields:
% \resemployer[Software Engineer]
% {\href{https://www.inspectorio.com/}{Inspectorio}}
% {Ho Chi Minh, Vietnam}
% \resjob[Developing API for internal web application.
% \begin{itemize}
%   \item Use Python to develop API for web application.
% \end{itemize}]
% {}{June 2019}{April 2019}

\resemployer[Software Engineer]{\href{https://www.larion.com/}{Larion}}
{Ho Chi Minh, Vietnam}
\resjob[Developing module to process data for a financial analysis firm.
\begin{itemize}
  \item Use \cplusplus{} to develop module that processes external data into
    internal formatted data.
\end{itemize}]
{Mars}{Apr. 2019}{Feb. 2019}

\resemployer[Software Engineer]{\href{https://www.luxoft.com/}{Luxoft Vietnam}}
{Ho Chi Minh, Vietnam}
% Indented entry for a job, providing title, start date, end date, and
% description fields:
\resjob[Developing HMI model/view for cars built by a Korean corporation.\\]
{Continental GmbH.'s DN8 dashboard}{Dec. 2018}{Aug. 2018}
\resjob[Maintain \cplusplus{} widget and animation layer for cars built by a Germany
corporation. See my job at Informa Solutions below.\\]
{Continental GmbH.'s MQB dashboard}{Aug. 2018}{Jan. 2018}

\resemployer[Software Engineer (on assignment from Luxoft Vietnam)]
{\href{http://www.informa.com.sg/}{Informa Solutions}}{Singapore}
\resjob[Maintain \cplusplus{} widget and animation layer for cars built by a Germany
corporation.
\begin{itemize}
  \item Add features into existing widget and animation upon request
  \item Corporate with model/view team to investigate and fix bug
  \item Make some small shell (run on git bash on Windows), and groovy (for
    Jenkins) scripts to automate the testing tasks.
  \item Technology: OSEK, \cplusplus{}, Microsoft Visual Studio, git, Jenkins
\end{itemize}
]{Continental GmbH's MQB dashboard}{Dec. 2017}{April 2017}

\resemployer[Software Engineer]{\href{https://www.luxoft.com/}{Luxoft Vietnam}}
{Ho Chi Minh, Vietnam}
\resjob[Mentoring Luxoft's new hire, who had just graduated from university.]
{Luxoft Internal Project}{March 2017}{April 2016}
\resjob[Develop HMI system for car built by a Korean corporation.
\begin{itemize}
  \item Joined in last phase of development
  \item Run system integration testing
  \item Fix minor bugs
  \item Used technology: \cplusplus{}, xml, Perforce p4
\end{itemize}
]{Harman-Becker}{March 2016}{Nov. 2015}

\resemployer[Software Engineer]{\href{https://www.fpt-software.com/}{FPT
Software}}{Ho Chi Minh, Vietnam}
\resjob[Develop software to process RTP stream into segmented video stream.
\begin{itemize}
  \item Used \cplusplus{}11 and boost to develop a system to process
    RTP stream which received from program provider (HBO, CNN,
    etc...) by blacking out, filtering advertisement scene (to
    replace it later), packetising, encrypting (with AES\Minus{}256),
    recording in real\Minus{}time to a format conformed with DASH
    and/or HLS standard.
  \item Technology: \cplusplus{}11, Boost, DASH, HLS, RHEL
\end{itemize}
]{DirecTV}{Oct. 2015}{Nov 2013}

\resheading{Other Development Experience}
\begin{itemize}
  \item Contributed minor improvement for git, dracut, notmuch, etc...
  \item Contributed to VoidLinux's build systems.
  \item Maintain some Void Linux's Packages.
\end{itemize}

\resheading{Education}
\begin{itemize}
  \item[]
    % The school entry provides name and location fields
    \resschool{FPT University}{Ho Chi Minh, Vietnam}
    % The degree entry is indented one level and provides a degree, major, date and
    % optional notes field.
    \resdegree[Graduated as good student]{B.Eng.}{Software Engineer}{Sept. 2013}
    % This is a two-column entry, indented one level and providing a text field and
    % date field.
    %        \ressubentry{Certificate, etc.}{Date}
\end{itemize}
%%\resheading{Academics}
%%    \begin{itemize}
% This provides a top-level bold item and a description field.
%%     \item[] \resdesc{Instructor}{Course 1, Course 2}
%%     \item[] \resdesc{Teaching Assistant}{Course 1}
%%     \item[] \resdesc{Referee}{Journal 1, Conference 1}
%%     \item[] \resdesc{Information Assurance Coursework}{Course 1, Course 2}
%% \end{itemize}

% This is a section for conferences.
%% \resheading{Conferences}
%% \begin{itemize}
%%     % Major conference entry: provides name and role fields
%%     \item[] \resconference{Major conference name}{Role (dates)}
%%     % Subconference entry (e.g. workshop, session, etc.): provides name and role
%%     % fields.
%%         \ressubconference{Workshop or session name}
%%         {Role (dates)}
%% \end{itemize}

% This is a section for publications
% In a future version, BiBTeX integration is probably a desired feature, but
% for now there is just a text entry type called resbib.
%% \footnotetext[1]{Denotes a peer-reviewed publication}
%% \resheading{Publications and Presentations}
%% \begin{itemize}
%% % This item provides article title (bolded) and rest-of-citation fields.
%% \item[] \resbib{``Article title''\footnotemark[1]}{Paper at Conference, Year.
%%                 Authors.}
%% \end{itemize}
\end{document}
